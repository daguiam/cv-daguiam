 % !TEX program = xelatex
%%%%%%%%%%%%%%%%%
% This is an sample CV template created using altacv.cls
% (v1.1.3, 30 April 2017) written by LianTze Lim (liantze@gmail.com). Now compiles with pdfLaTeX, XeLaTeX and LuaLaTeX.
% 
%% It may be distributed and/or modified under the
%% conditions of the LaTeX Project Public License, either version 1.3
%% of this license or (at your option) any later version.
%% The latest version of this license is in
%%    http://www.latex-project.org/lppl.txt
%% and version 1.3 or later is part of all distributions of LaTeX
%% version 2003/12/01 or later.
%%%%%%%%%%%%%%%%

%% If you need to pass whatever options to xcolor
\PassOptionsToPackage{dvipsnames}{xcolor}

%% If you are using \orcid or academicons
%% icons, make sure you have the academicons 
%% option here, and compile with XeLaTeX
%% or LuaLaTeX.
% \documentclass[10pt,a4paper,academicons]{altacv}

%% Use the "normalphoto" option if you want a normal photo instead of cropped to a circle
% \documentclass[10pt,a4paper,normalphoto]{altacv}

\documentclass[10pt,a4paper]{altacv}

%% AltaCV uses the fontawesome and academicon fonts
%% and packages. 
%% See texdoc.net/pkg/fontawecome and http://texdoc.net/pkg/academicons for full list of symbols.
%% 
%% Compile with LuaLaTeX for best results. If you
%% want to use XeLaTeX, you may need to install
%% Academicons.ttf in your operating system's font 
%% folder.


% Change the page layout if you need to
\geometry{left=1cm,right=9cm,marginparwidth=6.8cm,marginparsep=1.2cm,top=1.25cm,bottom=1.25cm,footskip=2\baselineskip}

% Change the font if you want to.

% If using pdflatex:
\usepackage[utf8]{inputenc}
\usepackage[T1]{fontenc}
\usepackage[default]{lato}

\usepackage{biblatex}

% If using xelatex or lualatex:
% \setmainfont{Lato}

% Change the colours if you want to
\definecolor{Mulberry}{HTML}{72243D}
\definecolor{AccentBlue}{HTML}{005b96}
\definecolor{HeadingBlue}{HTML}{011f4b}
\definecolor{SlateGrey}{HTML}{2E2E2E}
\definecolor{LightGrey}{HTML}{666666}
\colorlet{heading}{HeadingBlue}
\colorlet{accent}{AccentBlue}
\colorlet{emphasis}{SlateGrey}
\colorlet{body}{LightGrey}



% Change the bullets for itemize and rating marker
% for \cvskill if you want to
\renewcommand{\itemmarker}{{\small\textbullet}}
\renewcommand{\ratingmarker}{\faCircle}

%% sample.bib contains your publications
% \addbibresource{sample.bib}
\addbibresource{publications.bib}

\begin{document}
\name{Diogo Aguiam}
\tagline{Systems Integration Engineer}
\photo{2.8cm}{images/daguiam2016.jpg}
\personalinfo{%
  % Not all of these are required!
  % You can add your own with \printinfo{symbol}{detail}
  \email{diogo.aguiam@tecnico.ulisboa.pt}\par
  \phone{+351 916126735}\par
  % \medskip
  % \mailaddress{Address, Street, 00000 County}
  \location{Lisbon, Portugal} \par

}
\socialinfo{
  \linkedin{linkedin.com/in/diogoaguiam} \par
  \homepage{daguiam.github.io} \par
  \github{github.com/daguiam} \par
  \twitter{@diogoaguiam} \par
  % \orcid{kasfkha}
  %% You MUST add the academicons option to \documentclass, then compile with LuaLaTeX or XeLaTeX, if you want to use \orcid or other academicons commands.
  % \orcid{orcid.org/0000-0000-0000-0000}
}

%% Make the header extend all the way to the right, if you want. 
\begin{fullwidth}
\makecvheader
\end{fullwidth}

%% Provide the file name containing the sidebar contents as an optional parameter to \cvsection.
%% You can always just use \marginpar{...} if you do
%% not need to align the top of the contents to any
%% \cvsection title in the "main" bar.
\cvsection[page1sidebar]{Experience}

\cvevent{Microwave Diagnostics Developer - Invited Researcher}{Max-Planck-Institut für Plasmaphysik}{November 2015 -- Ongoing}{Garching, Germany}
\begin{itemize}
\item Installation and commissioning of microwave reflectometry diagnostics for Nuclear Fusion research
\item Development of density profile reconstruction codes for distributed computing cluster
\end{itemize}

\divider

\cvevent{Research Engineer, PhD Student}{Instituto de Plasmas e Fusão Nuclear}{February 2013 -- Ongoing}{Lisbon, Portugal}
\begin{itemize}
\item Design and assembly of RF electronics systems
\item Testing and validation of new microwave diagnostics
\end{itemize}
\divider

\cvevent{LabVIEW Developer}{Department of Physics, Instituto Superior Técnico}{February 2012 - July 2012}{Oeiras, Portugal}
\begin{itemize}
\item Development of LabVIEW interfaces for physics laboratories
\end{itemize}

\divider

\cvevent{Ciência Viva Monitor}{INESC-ID Lisboa}{June 2011}{Oeiras, Portugal}
\begin{itemize}
\item Teaching electronics and PCB fabrication to high school students
\end{itemize}
\divider




% \cvsection{Projects}

% \cvevent{Project 1}{Funding agency/institution}{}{}
% \begin{itemize}
% \item Details
% \end{itemize}
% \divider

% \cvevent{Project 2}{Funding agency/institution}{Project duration}{}
% A short abstract would also work.

% \medskip



\cvsection{Training}


\cvevent{Radiation Safety Instruction}{Max-Planck-Institut für Plasmaphysik}{2015, 2016, 2017}{Garching, Germany}
\begin{itemize}
\item Working in a nuclear fusion research facility including in-vessel access
\end{itemize}
\divider


\cvevent{Nanotechnologies - Athens Programme}{École Nationale Supérieure de Techniques Avancées}{November 2011}{Paris, France}
\begin{itemize}
  \item Fundamentals of nano-structures, nanophotonics, spintronics and microfabrication techniques
\end{itemize}
% \begin{itemize}
% \item 
% \end{itemize}
% \divider




% \cvsection{A Day of My Life}

% % Adapted from @Jake's answer from http://tex.stackexchange.com/a/82729/226
% % \wheelchart{outer radius}{inner radius}{
% % comma-separated list of value/text width/color/detail}
% \wheelchart{1.5cm}{0.5cm}{%
%   6/8em/accent!30/{Sleep,\\beautiful sleep}, 
%   3/8em/accent!40/Hopeful novelist by night,
%   8/8em/accent!60/Daytime job,
%   2/10em/accent/Sports and relaxation,
%   5/6em/accent!20/Spending time with family
% }

\clearpage
% \cvsection[page2sidebar]{Publications}
\newgeometry{margin=1.2cm}
\cvsection{Publications}
\nocite{*}

% \bibliographystyle{plain}
% \printbibliography
\printbibliography[heading=pubtype,title={\printinfo{\faBook}{Books}},type=book,sorting=ydnt]


\divider

% \begingroup
% \newrefcontext[sorting=ydnt]  

\printbibliography[heading=pubtype,title={\printinfo{\faFileTextO}{Journal Articles}},type=article,sorting=ydnt]

% \endgroup 
\divider


\printbibliography[heading=pubtype,title={\printinfo{\faGroup}{Conference Proceedings}},type=inproceedings,sorting=ydnt]

%% If the NEXT page doesn't start with a \cvsection but you'd
%% still like to add a sidebar, then use this command on THIS
%% page to add it. The optional argument lets you pull up the 
%% sidebar a bit so that it looks aligned with the top of the
%% main column.
% \addnextpagesidebar[-1ex]{page3sidebar}

\end{document}
